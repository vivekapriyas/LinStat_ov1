First we will show that $\frac{1}{n}\textbf{1}^T C = \textbf{0}^T$. 

$$\frac{1}{n}\textbf{1}^T C = \frac{1}{n} \textbf{1}^T \left(I - \frac{1}{n} \textbf{1}\textbf{1}^T\right) = \frac{1}{n} \textbf{1}^T - \frac{1}{n^2} \textbf{1}^T \textbf{1}\textbf{1}^T = \frac{1}{n} \textbf{1}^T - \frac{1}{n} \textbf{1}^T = \textbf{0}^T$$

where we have used that $\textbf{1}^T \textbf{1} = n$.

If we define $A = \frac{1}{n} \textbf{1}^T$, then we can say that $A \textbf{X}$ and $C \textbf{X}$ are independent if $A \Sigma C^T = \textbf{0}^T$. From the calculations above we have that $\frac{1}{n}\textbf{1}^T C = 0$ and since $A\Sigma C =  \frac{1}{n} \textbf{1}^T \sigma^2 I C = \sigma^2 \frac{1}{n} \textbf{1}^TC = \sigma^2 \textbf{0}^T = \textbf{0}^T$, are $\frac{1}{n} \textbf{1}^T \textbf{X}$ and $C\textbf{X}$ are independent. 


%Using that both the centering and the identity matrix are symmetric idempotent one get
%$$ 
%A \Sigma C^T = A \Sigma C =  \frac{1}{n} \textbf{1}^T \sigma^2 I (I - \frac{1}{n} \textbf{1}\textbf{1}^T) = \frac{ \sigma^2}{n} \textbf{1}^T I^2 - \frac{\sigma^2}{n^2}  \textbf{1}^T  \textbf{1}\textbf{1}^T I =  \frac{ \sigma^2}{n} \textbf{1}^T I - \frac{ \sigma^2}{n} \textbf{1}^T I = \textbf{0}^T.
%$$
%So, $\frac{1}{n} \textbf{1}^T \textbf{X}$ and $C\textbf{X}$ are independent, which can also been seen from what we showed above, namely that $\frac{1}{n} \textbf{1}^T C = \textbf{0}^T $.


Furthermore, since $\bar{X}$ and  $C\textbf{X}$ are independent and$S^2 = \frac{1}{n-1}(C\textbf{X})^T (C\textbf{X})$, i.e. $S^2$ is a linear combination of $C\textbf{X}$, we have that $\bar{X}$ and $S^2$ are independent. 

