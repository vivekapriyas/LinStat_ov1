Let $\text{E_b} = \{ x \in \mathbb{R}^2 | (\mathbf{x}- \mathbf{\mu})^T\Sigma^{-1} (\mathbf{x}- \mathbf{\mu}) = b \}$ be the ellipsoid shown in the figure which is centered in $\mathbf{\mu} = \text{E}[X]$. Applying Theorem 2.7 in Härdle and Simar, we can determine that the principal axes of $E_b$ will be in the direction of the eigenvectors of $\Sigma^{-1}$ and that their half-lengths will be given by $\sqrt{\frac{b}{\lambda_i}}, i =1,2$ where $\lambda_i$ are the eigenvectors of $\Sigma^{-1}$.  

We know that $Sigma$ and $Sigma^{-1}$ share the same eigenvectors and that the eigenvalues of $\Sigma^{-1}$ are  $\Tilde{\lambda_i}= \frac{1}{{\lambda_i}}$ where $\lambda_i$ are the eigenvalues of $\Sigma$. Thus, the principal axes of $\text{E_b}$ lie along $[1 1]^T$ and $[1 -1]^T$ and the lengths of the half-lengths are $d_1 = \sqrt{\frac{b}{4}} =$ and $d_2 =\sqrt{\frac{b}{2}}$ %%%jeg lurer på om det burde vært b^2 i ligningen for ellipsen? og vi må lage figur her og legge inn, derfor jeg har brukt navn d1 og d2

Additionally we, using the Mahalanobis transformation, recognize that 
\begin{equation*}
    (\mathbf{x}- \mathbf{\mu})^T\Sigma^{-1} (\mathbf{x}- \mathbf{\mu})
    = (\mathbf{x}- \mathbf{\mu})^T(\Sigma^{-1/2})^T\Sigma^{-1/2}((\mathbf{x}- \mathbf{\mu}) = Z^TZ
\end{equation*}

where $Z \sim N(\Vec{0}, \sigma^2 I $

- uttrykket = b er chi-sq med 2 frihetsgrader siden X er 2-dim, reff teorem 4.7, prøve å vise dette. 
- sentrert i mu.
- b er slik at p(X <= b) = 0.9 valg her, dermed er sannsynligheten for at X ligger i ellipsen 90\%. 
- halvaksene ... $\sqrt{(d^2 * \lambda_i}$) 


-fra frivillig øving: 

- 
We observe that $(\textbf{x}- {\mu})^T \Sigma^{-1} (\textbf{x}-{\mu})=b$ describes an an ellipsoid centered in $\mu = [0, 2]^T$. 

The half axes vil ha retningen til egenvektorene og 
egenvektorene vil være de samme for sigma og sigma inverse 

-HALFAXES
4
2.8 = 2\sqrt{2}

inverse of sigma:
- samme egenvektorer.

By performing a Mahalanobis transformation on 

transformasjonen tilsvarer en rotasjon?
nei tar bort egenskapene til mu og sigma slik at vi får enhetssirkel..

$(\textbf{x}- {\mu})^T \Sigma^{-1} (\textbf{x}-{\mu})=b$ %bold mu!!!

From this one can conclude that b is chi-squared distributed with two degrees of freedom. So the chosen value $b=4.6$ implies that the probability that $X$ falls within the given ellipse is $90\%$. 