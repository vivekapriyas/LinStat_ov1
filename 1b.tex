- uttrykket = b er chi-sq med 2 frihetsgrader siden X er 2-dim, reff teorem 4.7, prøve å vise dette. 
- sentrert i mu.
- b er slik at p(X <= b) = 0.9 valg her, dermed er sannsynligheten for at X ligger i ellipsen 90\%. 
- halvaksene ... sqrt(d^2 * \lambda_i) 


-fra frivillig øving: 

- 

The figure illustrates an ellipse centered in $\mu = [0, 2]^T$. 

By performing a Mahalanibis transformation on 

$(\textbf{x}- {\mu})^T \Sigma^{-1} (\textbf{x}-{\mu})=b$ %bold mu!!!

From this one can conclude that b is chi-squared distributed with two degrees of freedom. So the chosen value $b=4.6$ implies that the probability that $X$ falls within the given ellipse is $90\%$. 