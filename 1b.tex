- uttrykket = b er chi-sq med 2 frihetsgrader siden X er 2-dim, reff teorem 4.7, prøve å vise dette. 
- sentrert i mu.
- b er slik at p(X <= b) = 0.9 valg her, dermed er sannsynligheten for at X ligger i ellipsen 90\%. 
- halvaksene ... sqrt(d^2 * \lambda_i) 


-fra frivillig øving: 

- 
We observe that $(\textbf{x}- {\mu})^T \Sigma^{-1} (\textbf{x}-{\mu})=b$ describes an an ellipsoid centered in $\mu = [0, 2]^T$. 

The half axes vil ha retningen til egenvektorene og 
egenvektorene vil være de samme for sigma og sigma inverse 

-HALFAXES
4
2.8 = 2\sqrt{2}

inverse of sigma:
- samme egenvektorer.

By performing a Mahalanobis transformation on 

transformasjonen tilsvarer en rotasjon?
nei tar bort egenskapene til mu og sigma slik at vi får enhetssirkel..

$(\textbf{x}- {\mu})^T \Sigma^{-1} (\textbf{x}-{\mu})=b$ %bold mu!!!

From this one can conclude that b is chi-squared distributed with two degrees of freedom. So the chosen value $b=4.6$ implies that the probability that $X$ falls within the given ellipse is $90\%$. 